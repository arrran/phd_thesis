\newpage
\TOCadd{Abstract}

%\vspace{10mm}


\begin{center}
\textbf{ABSTRACT}
\end{center}
Channels melted into the base of ice shelves are thought to influence ice shelf evolution by redistributing melt patterns and reducing their structural integrity. Theories of basal melt and fresh water plume processes which occur in channels are generally poorly constrained by observations.
At the grounding line of the Kamb Ice Stream, Antarctica, a distinctive valley in the ice surface topography reveals the location of a subglacial channel.
This thesis uses new observations of the ice shelf basal channel to describe processes which cause the formation and growth of the channel. 
Using a combination of detailed ground--based observations, remote sensing, and interpolation we map the surface and basal topography of the area.
The basal channel is observed to have incised up to 50\% of the ice thickness and extends 6 km inland from the previously estimated grounding line of the stagnant Kamb Ice Stream.
Remote sensing products show that the channel has grown upstream over time, and likely continues to grow. Modern surface lowering at the steep upstream inception of the channel reveals a region of focused melt where a subglacial outlet meets the ocean cavity. We estimate this basal melt to be at least 20 m/yr in a narrow (200 m x 1.5 km) zone. Downstream from the melt region, repeat phase sensitive radar observations reveal a large region of basal marine ice and accretion. A year--long time series of phase sensitive radar observations shows that accretion is generally consistent in magnitude, though oscillates in strength at tidal periods.
Using the channel shape mapped by radar surveys as a key constraint, we model ocean circulation and basal ice melt with the MIT Global Circulation Model. The model predicts that ocean circulation in the channel is driven by a melt water plume coupled with  estuarine--like dynamics.
Predicted melt is strongest at the inception of the channel where the plume ascends the positive gradient of the ice base. The plume causes a strong downstream flow of fresher water in the upper half of the water column, which overlies an upstream flow of salty bottom water from the Ross Ice Shelf ocean cavity. The inflow of salt is entrained into the plume and provides energy to melt the ice base.
Higher melt rates, up to 17 m/yr, were only modelled when strong intermittent subglacial drainage flushed the channel and was replaced by salty ocean water in quiescent times. 
We conclude that the channel is likely formed by an upstream migrating meltwater plume triggered by episodic subglacial drainage. 


% Over--snow radar imaging constrains a steep inception, and shows that not all of the basal shape is manifest at the surface. Three wide sections of the channel correspond to bends, and are expressed on the subaerial ice surface as basins.



