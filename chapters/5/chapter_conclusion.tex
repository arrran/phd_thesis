
%%
%%

%%data 32 
%% apres 19
%%ocean 35
%% total 86

%% lauren 79 wei ji 102
\startchapter{Synthesis}



% \tableofcontents

% \listoffigures

In this chapter, we first  discuss key findings from the thesis, links between these findings, and their implications (Section \ref{sec:findings}). Additionally, we outline discrepancies in results and suggest ideas for future work which could follow from the research presented.
In Section \ref{sec:unified}  we summarise and unify the theoretical models presented in this thesis. This unified model is the theoretical description of the channel and processes associated with the channel which are most likely, based on results from the thesis and existing theories.

% Providing detailed observations of the ice shelf basal channel contributes a case study for modelling experiments as well as comparison for research seeking to describe processes which occur in ice shelf basal channels.

\section{Findings and Implications} \label{sec:findings}

% The goal of this thesis is to improve our ability to understand and predict ice ocean interaction both in the channel and in the surrounding ice and ocean cavity. I aim to narrow the possibilities for potential conditions and processes within the channel, constraining processes upstream in the ice stream as well as downstream in the ocean cavity. Processes include the impact of subglacial discharge and ocean dynamics on basal melting of ice, and the impact of ice topography on  subglacial drainage and ocean circulation. In addition, because this is among the first of a large body of work planned for the channel feature, a secondary goal of this research is to guide future research.

\subsection{Steep subglacially sourced slume}

%subglacially sourced
The basal channel was most likely incised by a subglacially sourced meltwater plume (discussed in Section \ref{sec:steep}). This is supported by evidence that the channel's location corresponds to a  subglacial discharge outlet (\cite{alley2016impacts,le2009subglacial}, Figure \ref{fig:fieldwork_location}).
Describing melt in the channel as driven by a subglacial plume (as described by \cite{jenkins1991one}) is consistent with observations of the steep channel shape, and localised active surface lowering.

In Section \ref{sec:surfacetopog} the temporal gradient in ice surface topography was mapped across the study area with remote sensing data (Figure \ref{fig:4square_channel} D). This revealed an active region of surface lowering at the channel inception, which was likely the surface expression of ongoing subglacial melt (Figure \ref{fig:4square_channel} D).  
Finding active localised surface lowering was instrumental in better understanding modern processes in the basal channel, and informed our theories of conceptual models of plume dynamics in the channel, as well as ocean modelling experiments that described ocean circulation in the channel (Chapter Four). 
% In this section we discuss how this finding linked with other results, and how these helped form a conceptual model.

The region of surface lowering is focused around a small area at the inception of the channel. We attribute the surface lowering to basal melt from a meltwater plume that ascends the sloped basal channel. Radar surveys showed that the basal channel is steep at its inception (Figure \ref{fig:thickness_surfacecolour}). A meltwater plume likely flows downstream up the gradient of the ceiling of the channel. As the plume ascends it melts ice, becoming more fresh and buoyant. The upward plume drives circulation, entraining warm salty water from the ocean depths to the ice face, enhancing melt.  Plume theory predicts that melt is strongest where a plume starts to ascend and decreases in strength as it continues to ascend \citep{jenkins1991one}. We suggest that because the channel is steep at its inception the meltwater plume does not travel far downstream over its ascent, resulting in the small horizontal area of melt that we observe. 

We suggest that a steep plume developed through feedbacks as described by \cite{sergienko2013basal}, and in Section \ref{sec:data_intro}. Such feedbacks between a steepening slope and plume melt will continue until basal slope, and the plume flow is vertical (described in more detail in Section \ref{sec:steep}).    
Future work could solve the plume model presented in \cite{jenkins2011convection}, adapted to have a time--evolving domain shape. The shape resulting from plume melt in the absence of advection could be compared to the observed shape of the channel. 

In Chapter Four, ocean modelling results using MIT Global Circulation Model (MITgcm) showed the strongest melt in the channel domain was consistently at the inception of the channel (Figure \ref{fig:surge_melt}). This independently supports our theory that the steep channel and localised surface lowering are caused by melt from a plume and shows that the MITgcm model setup is successfully modelling some plume processes.
Because modelled melt was focused over just a few grid cells at the inception of the channel,  MITgcm likely failed to reproduce plume processes accurately (Section \ref{ocean_discuss}). The channel would be best modelled in future with an adaptive grid with very high resolution at the inception of the channel and modifications to the melt model to allow for melting of vertical cell walls (as discussed in Section \ref{ocean_discuss}).

\subsection{Intermittent upstream channel migration}

In Chapter Two, satellite imagery showed that the channel has migrated upstream in the last 40 years (Figure \ref{fig:historic}).  We theorise that the focused plume melt location had moved upstream over time to form the length of the channel (Section \ref{sec:discussion}). Topographic maps revealed multiple basins on the surface, which are the expressions of wide sections of the basal channel (B1-3, Figure \ref{fig:4square_channel}). We suggest that the oscillations in channel width were  caused by intermittent melt (Section \ref{sec:retreat}), and that wide sections of the channel correspond to regions where the plume was focused for more time.
Results from ocean modelling (Chapter Four) strongly support the theory that subglacial drainage is intermittent, and imply that episodic flow is an important part of the ice slope and plume melt feedback that forms the channel, in that a deep channel could only be incised with a strong plume which only develops after the channel is replaced with more saline water. Due to the long and skinny shape of the channel, a full overturning of water required some time (Section \ref{sec:surge_inflow}).  High melt rates, closer to those estimated in Chapter Two, were only modelled with strong intermittent drainage which flushed the channel and was replaced by salty ocean water in quiescent times. 
Melt was only found along the apex of the channel at the beginning of drainage events, after which, accretion would occur at the apex. This corresponds to ApRES observations from Chapter Two which showed accretion at the apex of the channel on a base of marine ice, suggesting that the plume was not strong during the period. 

We outline two constraints on the time scale of intermittency of the plume. Firstly, using a continuously recording ApRES radar we chronicled changes at the ice base over 2020 (Chapter Three). The ApRES time series showed a relatively stable basal mass balance over one year. Assuming that the consistency in mass balance was caused by the same plume dynamics that caused melt at the inception of the channel shown by surface lowering, we infer that melt at the inception of the channel was steady for a year-long period. Secondly, in Section \ref{sec:surface_lowering} the active surface lowering was shown to be similar when calculated over 2012-2016 and over 2019-2020, suggesting that the melt plume was consistent over the entire period from 2012 to 2020. This suggests that the theorised episodic nature of melt can have long periods ($>$10 years) of stability.

Future research can improve our understanding of the channel's formation by monitoring the channel growth with modern, high--resolution remote sensing products. With a few more decades of observations, the rate of inland migration of the channel and the degree of variability in the rate will likely become clear.

\subsection{Ocean connection}

Plume driven melt in the channel was found to be strongly dependent on the availability of saltwater (Section \ref{ocean_discuss}).   A stronger connection to the larger Ross Ice Shelf cavity increased basal ablation rates.  Ocean exchange increased the salinity in the channel through the  inflow of salty bottom water (as described by \cite{goldberg2019accurately}), as well as more outgoing fresh water. In Chapter Four, model results predicted that two layers of water develop in the cavity similar to estuarine flow, with fresh downstream flow on the upper layer and salty upstream flow on the lower layer. This flow pattern is conducive to replacing fresh cavity water with salty ocean water. 

Ocean modelling (Chapter Four) revealed a relationship between the strength of the ocean connection and the changing width of the channel (described in Chapter Two). The strong estuarine circulation with outflow layered on top of inflow was shown to break down in wide sections of the channel, where crossflow circulation cells developed (Section \ref{ocean_discuss}). Therefore, we expect the existence of wide sections of the channel to slow the replacement of seawater and reduce basal ablation rates at the plume.
The ApRES time series (Chapter Three) provided additional evidence of an active connection between the channel and the ocean, showing that the tide influences basal processes. Assuming that this accretion was caused by the same plume dynamics that cause melt at the inception of the channel shown by surface lowering, we infer that the subglacial plume is influenced by tides. It follows that melt may be tidal at the channel inception.
This observation described a connection between the plume processes and ocean processes, and therefore provided evidence to support the modelling--based prediction that melt rates and plume processes are influenced by the channel's connection to the larger Ross Ice Shelf cavity. 

\subsection{Melt rate estimation}

In Chapter Two the ice base topography and the temporal gradient in ice surface topography were combined to estimate a lower bound of $\approx$ 20 m/yr on basal melt rate (Section \ref{sec:surface_lowering}). This bound was used as a key constraint when modelling ocean circulation in the channel in Chapter Four. 
In these model experiments, the domain shape was constrained by the ice base estimation from Chapter Two, and the key variable used in the experiment was water exchange in and out of the model domain, through the upstream and downstream boundaries. Model outputs included temperature, salinity, velocities and melt rates, the latter of which  were compared to the melt rate bounds estimated in Chapter Two. 
While the model meltwater plume expelled fresh water along the top of the channel, warm salty water flowed upstream at the bottom. In model runs, this upstream salt flux was necessary to drive significant melt rates.
The ocean circulation model did not reproduce the high basal melt rates ($\approx$ 20 m/yr) estimated in Chapter Two. We attribute this to the inability of the model to solve for a strong subglacial drainage at the upstream boundary, or the inability of the model to develop focused melt. While results from Chapter Two suggest melt was strongly focused on the inception of the channel, model outputs showed a more distributed melt pattern. This discrepancy may have been solved by extending the domain further upstream and increasing the horizontal resolution to more accurately resolve processes at the inception of the channel. 
Identifying and bounding modern melt processes highlights and guides opportunities for future work. For example, through the oceanographic observations taken in the 2021--2022 drilling project, future work can attempt to better constrain plume dynamics and subglacial drainage which cause the observed surface lowering.  

\subsection{Accretion spatial pattern}

Plume theory (as described by \cite{jenkins2011convection}) predicts melt to occur over a certain distance downstream from the origin of a plume. After that distance, the plume is predicted to lose energy and become supercooled, first ceasing to melt, then further downstream causing accretion.
Accretion downstream from the predicted plume was observed in ApRES basal surveys presented in Chapters Two and Three. Basal accretion or marine ice was found along two perpendicular tracks (Figure \ref{fig:APRES_melt}). The track along the apex of the basal channel showed  accretion on the edge of the region of surface lowering (Figure \ref{fig:geophysics_overview}, and \ref{fig:thickness_surfacecolour}). These results, supported by plume theory \citep{jenkins1991one}, suggest that melt likely only occurs in the channel as the plume ascends, and over a small distance downstream the plume starts to accrete ice. The surface expression (Figure \ref{fig:4square_channel}) does not show raising of ice immediately downstream of the melt, suggesting that the  melt signal dominated and hid the nearby accretion.
In Chapter Three an ApRES survey chronicled changes at the ice base over 2020 (Figure \ref{fig:alltimeseries}), downstream from the region of surface lowering. This identified basal marine ice over the whole period and found that the ice base was changing consistently, most likely through accretion (Section \ref{sec:apres_discussion}). These results showed that melt does not occur down the length of the channel apex, confirming that melt is concentrated at the inception of the channel for a period of one year.
This aligns with modelling results in Chapter Four which showed accretion at the apex of the channel downstream from the inception (Figure \ref{fig:accretion}).  Rather than showing a supercooled plume, modelling results gave an alternative explanation for conditions that cause accretion, predicting that accretion would occur when pockets of fresh water developed on the ceiling of the channel. These buoyant pockets of meltwater were stable and therefore isolated from the cavity circulation. Water in the stable pockets continued to cool and accreted ice. The formation of these isolated pockets of fresh water is highly dependent on bathymetry and relies on local highs in the ice ceiling (Figure \ref{fig:4square_channel}). 
The ice base map from Section \ref{sec:icebasetopog} showed slight highs in the estimated ice ceiling where we would likely see pooling of fresh water and accretion. The location of the ApRES time series (Figure \ref{fig:geophysics_overview}), was near the centre of the basin B1 (Figure \ref{fig:thickness_surfacecolour}) which was estimated to correspond to a local high in the ice base, suggesting that the observed marine ice could be forming in a pocket of trapped fresh water.


\subsection{Surface and basal topography}

The basal and subaerial surface topography around the channel were mapped using radio echo sounding and remote sensing respectively (Chapter Two).  
These results were instrumental in directing subsequent research. Firstly, we used the basal map as a basis for modelling experiments in Chapter Four. Secondly, the map was used to direct field research, including two field campaigns to the channel. In the 2020--2021 field season an ApRES was installed at a location guided by the ice base map, at the intersection of the apex of the channel (Figure \ref{fig:geophysics_overview}) and the region of surface lowering (Figure \ref{fig:4square_channel} D). In the 2021--2022 field season, the ice base map guided drilling through the ice to the channel cavity, where a large range of observations were made including oceanographic observations, sediment cores and a submarine survey \citep{horgan2022channel}. The ice base map will continue to guide future remote sensing work or any future field studies of the channel. Knowledge of the basal shape will allow surface observations to be better linked to basal channel processes. 
We recommend that future studies revisit the radar data used in Section \ref{sec:radar} using techniques to attempt to map the properties of basal ice as attempted in \cite{macgregor2011grounding}. Using this radar data, it may be possible to reveal areas of marine ice or spatial changes in the ice base which could inform models of plume dynamics.

\subsection{Bridging}

Observations of surface and basal topography in Chapter Two displayed that the surface valley is not a direct reflection of the basal channel, and bridging stresses cause basal features to be smoothed or without surface expression. In particular, the basal channel extends further upstream than the surface valley, and the surface valley is wider than the basal channel. The surface expression of basal melt is similarly smoothed to cover an area wider than the channel, and overlaps observed accretion. 
We recommend that future work develops better inverse techniques aimed at constraining the ice base based on the surface expression. Basal melt could be better constrained by using physical models of ice (e.g. with a full-stokes finite element model such as Elmer--Ice \cite{gagliardini2013capabilities}) to model the response of the subaerial ice surface to basal melt (similar to an experiment by \cite{drews2020atmospheric}). Coupled with an analysis of modern remote sensing products, inverse models could be calibrated with field observations of basal melt and with maps of the basal and surface topography.  While deriving basal melt from surface changes is well established over coarse resolutions using flux divergence techniques \citep{berger2017detecting}, few studies have sought greater than ice thickness scale resolutions \citep[e.g.][]{mankoff2012role}. With a wealth of remote sensing data of increasingly good resolution, better resolving basal ice shelf micro--topography would lead to stronger predictions of the circulation of water under ice shelves due to the fact circulation is strongly influenced by basal topography \citep{holland2003ice}. 

\subsection{Inconsistencies}

\subsubsection{Melt pattern}

Modelling experiments in Chapter Four revealed that due to the high salinity of bottom water, the channel generally experiences higher melt lower in the water column. This implies that the channel likely experiences melt along the walls, effectively widening the channel. While surface and basal observations from Chapter Two did not find consistent wall melt, the ApRES survey crossing the channel showed a similar trend with lower magnitudes of accretion at the sides of the channel relative to the centre (Figure \ref{fig:APRES_melt}). However, the ApRES cannot measure melt on steep walls due to off nadir reflections, which may also contribute inaccuracies to lateral trends in basal mass balance.

In Chapter Two we discussed surface observations which suggest that the channel was migrating to the left, similar to a channel described by \cite{chartrand2020basal}. We attributed this movement to the asymmetry of circulation caused by the Coriolis force. This was not supported by modelling experiments in Chapter Four which showed the Coriolis force had little effect on circulation in the channel and did not show a lateral (left--right) bias in melt and accretion. Additionally, in Chapter Two we predicted that ledges were growing in the channel through accretion, similar to a channel described by \cite{dutrieux2014basal}. However, this result was again not supported by modelling results which showed accretion only occurring in the top section of the water column, where water was more fresh.

There is an overall trend in the discrepancies between observations and the modelling results. Generally, modelling results showed the most variability in melt processes laterally and showed less variability in the downstream direction. On the other hand, observations imply there is more downstream variability in melting.
We suggest that much of these discrepancies are caused by the smoothing of the expression of basal mass balance at the surface. Because the channel is not very wide, lateral variability will hardly be expressed at the surface. The lack of downstream variability in modelling experiments may be due to the lack of resolution to resolve plume dynamics at the inception of the channel.

\subsubsection{Accretion rates}

The accretion rates of $\approx$ 0.8 m/yr presented in Chapter Three are different to the accretion rates $\approx$ 3 m/yr presented in Chapter Two, the former of which is more accurate. Accretion rates in Chapter Two were calculated from repeat surveys one--year apart, over which internal reflectors had moved too much to track accurately. This resulted in an inaccurate strain estimate. The ApRES transect across the channel (Figure \ref{fig:APRES_melt} A) was surveyed on 07-12-2019, 21-12-2019, and 23-12-2020, whereas the transect along the channel apex (Figure \ref{fig:APRES_melt} A) was surveyed only on the latter two dates. The comparison across all ApRES repeat surveys in Chapter Two was therefore made over the one year time frame for consistency.  The inaccuracies of the accretion rates do not affect any other results or conclusions made for two reasons. Firstly, when compared to rates calculated over 20 days, the cross-channel transect shows accretion rates $\approx$ 0.8 m/yr, but has the same pattern of relative accretion to that displayed in Figure \ref{fig:APRES_melt} A.  Secondly, because the magnitude of accretion is likely not representative of the actual accretion due to interference from multiple reflectors, no conclusions were made based on absolute accretion rates.

\subsection{Implications}

This thesis was the initial detailed exploration of the sub--ice--shelf channel, which is the subject of a much larger body of research. The main implications of this thesis are to guide future research, outlined above. Additionally, our observations and constraints, such as the melt rate bounds, channel shape, or theories of plume migration can be used as a case study to further constrain theories of channel formation and growth. Future attempts to model ice shelf channels can use these observations as constraints. 

\section{Unified theoretical model} \label{sec:unified}
% \section{Unified theory of everything}

Based on the results presented in this thesis, we have theorised a description of the channel and processes associated with the channel.  In plain terms, we have tried to answer the fundamental questions `what is it?' and `how does it work?' In this section, we present a unified theory describing the channel and associated processes based on results and literature.  References will be provided to parts of the thesis which explain theories in more detail, including the likely hood and other potential theories.

\subsection{Description of the Present Channel}
The basal channel shape is described by the ice base map (Figure \ref{fig:4square_channel}). An initial slope between 0-45 degrees rises to the highest part of the channel apex (Figure \ref{fig:thickness_surfacecolour}). This upstream part of the channel is very narrow, so has little or no surface expression as it is supported by bridging stresses (Section \ref{sec:bridging}). The apex of the channel decreases in height downstream, after which it follows for 8 km at a consistent height.
The basal channel has three bends. At each bend, the channel is wider (Figure \ref{fig:4square_channel}).
The surface expression of the basal channel (surface valley) starts downstream from the inception of the basal channel (Section \ref{fig:4square_channel}). The surface valley is wider than the basal channel as it is smoothed by the lateral distribution of stress in the ice. Wide parts of the channel provide less buoyancy so cause slumps in the ice, and so correspond to basins on the surface (Section \ref{sec:bridging}).
Melt is focused at the inception of the channel over a small area, as shown by localised surface lowering (Figure \ref{fig:4square_channel}), the steep channel inception (Figure \ref{fig:thickness_surfacecolour}), and ApRES observations (Figure \ref{fig:APRES_melt}). Due to bridging stresses in the ice, this melt appears as a larger region extending around the channel, and does not appear directly over the narrow channel tip  (Figure \ref{fig:4square_channel} D). 
The observed melt is driven by the supply of salt from the larger Ross Ice Shelf cavity (Section \ref{ocean_discuss}), to a meltwater plume at the inception of the channel. A circulation cell dominates channel circulation, with upstream movement of warm salty water from the ocean at the bottom, and downstream flow of cooler fresh meltwater above this layer (Figure \ref{fig:medium_flow_circulation}). 
Downstream from the melt region, accretion occurs (Figure \ref{fig:APRES_melt}).
Accretion is especially strong on the true right of the channel, where surface observations reveal surface raising coincident with ledges in the ice base (Figure \ref{fig:4square_channel} D). 

\subsection{Formation}
The formation of the channel was coincident with the retreat of the Kamb Ice Stream grounding line to its present location (Section \ref{sec:retreat}). Effective pressure at the base at the grounding line is zero, meaning the creep closure of the drainage channel no longer counteracted the melting of the channel walls, and it grew unabated (as described by \cite{drews2015evolution}). 
When the grounding line initially retreated, there was a subglacial drainage outlet where B3 is currently situated (Figure \ref{fig:thickness_surfacecolour}). Buoyant subglacial drainage met the ocean triggering a meltwater plume which incised into the ice shelf above (as described by \cite{hewitt2020subglacial}). Melt rates were strongest at the bottom of the plume, closer to the source of entrainment and saltier bottom water, and as a result, the downstream cross--sectional shape developed a more negative change in gradient downstream until the incised channel was steep above B3 (Section \ref{sec:steep}). The plume continued to melt the steep ice wall above the channel and incised into the ice upstream creating an embayment, which allowed for salty water to flow up the channel at the seafloor, providing energy to the plume for melt (Section \ref{ocean_discuss}). 

When large pulses of subglacial meltwater from a flood upstream drained (Section \ref{sec:retreat}), the plume ascended higher, and the channel melted higher \ref{sec:ocean_pulse}. When there was less meltwater, the plume melted the walls at a lower elevation, widening the channel. The widening and narrowing of the channel correspond to periods of low and high discharge rates respectively (Section \ref{sec:ocean_pulse}).
% \section{Future work}
% Future work on the channel should use the channel as a model to ratify three dimensional ocean plume models, confirm basal melt models and quantify subglacial drainage in the region.
% Lastly, to assist all research in the area, local high resolution remote sensing products should be developed, including a more accurate, higher resolution velocity and strain maps, a detailed local grounding line, and modern digital elevation models of the subaerial ice surface.
% Thirdly, a sweep of equipment could be deployed to the field location. With direct access through a borehole more oceanographic moorings would reveal valuable information about ocean dynamics in the channel. Ideally, surveys would be spatially distributed varying from a submarine like Icefin. This could capture the ice base topography and bathymetry, and varying ocean conditions in the channel. Surveying the inception of the channel would be especially useful for constraining subglacial drainage and plume processes.
% Even without direct access, surveys similar to those presented in this thesis, but at higher resolution would be useful. This could help to better constrain the inception shape of the channel, an array of apres could spatially map changes in the base, and maybe even the upstream migration of the channel. Mapping the channel downstream would better constrain interaction between channel and ocean, mapping the features downstream will help figure out what they are.